\addcontentsline{toc}{part}{Abstract}

\chapter*{Abstract}
Fixed-pitch quadcopters has become increasingly popular over the past years and have become the platform of choice for researchers and hobbyists. The platform is preferred due to its mechanical simplicity and robustness compared to other flying platforms. \bigskip

Lately, there has been a significant reduction in cost related to quadcopters. The hardware has become cheaper, smaller and carries more functionality and processing power then ever before. This has made the technology more widely available and has contributed to a leap in advances and new applications. \bigskip

Beside all the research, advances in technology and the inherent simplicity, the fixed-pitch quadrotor has its limitations. A fixed-pitch quadrotor achieves stability and flight control by changing the speed of the individual motors by a increase or decrease in voltage over the motors. Changing the rotational speed of the motor is not an instantaneous process and requires time to accelerate. The motors has to overcome the inertia of the rotating parts to get the propellers spinning faster, or it has to dissipate the kinetic energy stored in the rotation to go slower. Flight control based solely on changing motor speeds works well in many applications, but has shown to be inadequate in turbulent and difficult flying conditions, especially when landing. \bigskip

The question, is it possible to improve the capabilities of quadcopters by implementation of variable pitch. With variable pitch actuation, thrust can be changed almost instantaneously only limited by the speed of the servo actuation. With variable pitch, the simple quadcopter may overcome its limitations of instability in challenging flying conditions and turbulent landings.

