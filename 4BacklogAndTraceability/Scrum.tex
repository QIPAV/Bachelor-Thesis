\chapter{Scrum}
Scrum is an iterative and incremental development framework that coincides with the agile philosophy. The requirements in Scrum are represented by User Stories. These stories are short descriptive texts written with the end-user in mind. Instead of using a traditional ranking system, Scrum re-prioritizes the tasks at each Sprint and whatever is on top of the Product Backlog is the most important. A short introduction to some of the important elements in Scrum will follow. 

\section{Backlog}
Scrum has two types of backlogs, the Product Backlog and the Sprint Backlog.\bigskip

\textbf{The Product Backlog} is a prioritized list of customer-centric features. The backlog contains everything that might be done in a project and the items in the backlog are called "User Stories". During the Sprint Planning meeting, the team decomposes User Stories into more manageable tasks and chores. The items in the Product Backlog should be small enough, so that a team member could accomplish several of them in two-week Sprint. The Product Backlog is prioritized from highest- to lowest priority where the topmost item is the one with the highest priority. \bigskip

\textbf{The Sprint Backlog} is a collection of the User Stories which the team plan to accomplish during the Sprint. This is normally created at the Sprint Planning meeting where team members can pick User Stories from the Product Backlog and bring it into the Sprint Backlog. The team decides how much work they are able to accomplish in the next Sprint. After the team members have selected the User Stories for the upcoming sprint, they are decomposed into tasks. These tasks are sorted as "to do", "in progress", "done" and "approved". As a general rule, one task should never exceed 3 days of work and should ideally be less than one days work.\bigskip 

The Product Backlog Items are estimated with relative units called story points. When estimation is done, the User Stories are rated relative to each other. Research has shown that rating in relative units will yield more precise time estimates than when trying to sum up all the hours for each task. 

\section{User Stories}
The Product Backlog Items (PBIs) are referred to as User Stories. These are features, functionality or "requirements" derived from what the customer wants or needs. The User Stories are often written from the perspective of the end user and with the customer in mind. The syntax is \textit{As a/an(user), I want/need, So that(purpose)}. The purpose of writing User Stories is to understand what the customer wants. Therefore, these Stories are often derived from conversations with the customer or stakeholders. Whereas other project models use a more standard way of representing requirements i.e the IEEE-830 Standard with \textit{"The system shall.."} - statements, Scrum utilizes User Stories. \bigskip

\textbf{Example of a User Story} \\
As a customer, I want the quadrotor to weigh less than 2.5kg. So that the quadcopter is in compliance with laws and regulations. 

\clearpage

\section{Acceptance Criteria}
In Scrum the most important user stories have Acceptance Criteria. It provides the criterion that has to be met for a user story to be assessed as complete.  
The syntax of these are: \textit{Given(some context), When(some action is carried out), Then(something expected happens).} \bigskip 

\textbf{An example of an Acceptance Criterion} \\
Given that we have a quadrotor, when the quadrotor is weighed, then the quadrotor shall weigh less than 2.5kg.





 