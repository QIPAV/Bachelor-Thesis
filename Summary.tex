%\addcontentsline{toc}{part}{Summary}
\chapter*{Summary}
% %Maks en halv-0,75 side

% - Hva skulle vi gjøre og hvorfor?
This thesis was conducted with the purpose of investigating variable pitch for use on quadcopters by appointment for FFI. The goal is to investigate the potential benefits addition of variable pitch can have on quadcopters. The objectives were to build two quadcopters with and without variable pitch, compare the properties between the two configurations and see if variable pitch gives improved responsiveness compared to fixed pitch.  \bigskip

%-  Hva har vi gjort og hvor langt kom vi?
Both a variable and a fixed pitch quadcopter have been built. The fixed pitch quadcopter did not work as intended and was discarded. Instead,  the variable pitch quadcopter was flown in fixed configuration for comparison.\bigskip 

% - Hvilke tester er utført?
Tests have been performed to determine the increase in capabilities of the quadcopter with variable pitch. Specifically tests have been carried out to examine the increase in responsiveness and precise landings. Additionally, a general analysis of variable pitch is given.\bigskip

The results show that....................................................
%% 




%Two multirotor helicopters has been developed throughout the project, one with variable pitch propellers and one with traditional fixed pitch propellers.  \\

%Two quadcopters have been built, with and without variable pitch. \\



% - Hva er resultatet? 


% - Hva er konklusjonen?




%This project has conducted research on variable pitch as motor configuration on quadcopters. 


%This thesis has explores the benefits and characteristics of variable pitch quadcopters for use on quadcopters. A comparison of traditional fixed pitch quadcopters with variable pitch has been conducted. 

\begin{comment}
The basis of this project is the hypotheses that addition of variable pitch helps the traditional quadcopter overcome its limitations. Traditional fixed pitch quadcopters are limited by the speed at which the propellers can change their rotational speed. The rate of change depends on the inertia of the rotating parts and the torque- speed characteristics of the motors. 
The responsiveness of fixed pitch quadcopter has shown to be inadequate in turbulent and challenging conditions where fast changes of thrust are required. \bigskip

The theory is that variable pitch gives the possibility of near instant thrust changes. Implementation of variable pitch increases the weight and mechanical complexity of the quadcopter over regular fixed pitch quadcopters. The question is, does the benefits of variable pitch outweigh the changes to the simple and robust fixed pitch quadcopter. \bigskip

Research and development of a variable pitch quadcopter has been carried out on the request from FFI. The objectives were to identify the benefits of using variable pitch over traditional fixed pitch quadcopters. The objectives are to investigate how variable pitch affects the responsiveness of the vehicles, how variable pitch improves flight stability, landings and ground effect. \bigskip

Initially, two quadcopter were supposed to be built in order to be compared to one another. The first fixed pitch quadcopter built, did never work as intended and was discarded. However, the variable pitch quadcopter was built and did work. The comparison between variable and fixed pitch quadcopters was instead done by fixating the pitch angle of the variable pitch quadcopter, testing it in fixed pitch mode. By using the same quadcopter for comparison, all other factors are kept constant except the addition of variable pitch.

In the data found, there were little evidence to support that variable pitch in itself gives more stable flight or landing. The most evident benefit of variable pitch according to the tests, is increased responsiveness.

A carbon quadcopter was designed and built, with a modified Arduino Nano micro-controller and an IMU for the flight controller.
\end{comment}