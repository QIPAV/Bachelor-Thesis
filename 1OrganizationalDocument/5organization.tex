\chapter{Organization and Tools}

This project follows the Scrum framework. See the project model document to learn more about Scrum (Appendix \textbf{\ref{app:projectmodel}}). A project plan has been made, represented by a Gantt-chart to give an overview and to ensure that the project deadline is met. 

\section{Sprints}

Scrum works by utilizing project cycles, which is a fixed-length iteration containing all the traditional project phases in one cycle. These cycles or iterations will be referred to as sprints.\\
\\
The sprints have planned start and ending every other Tuesday, starting February $17^{th}$ $2017$. 


\section{Work Hours}

The group has agreed that working hours are between 8:15 to 16:15, Tuesday to Friday. This gives a total of 32 planned work hours per person in one week. After Easter, the team will have one additional day, since exams are over. One day a week is a nonworking day, preferably Saturday. Sundays will be utilized if additional work days are required. 

\section{Meetings}

In this project, meetings are performed in conformance with the Scrum ceremonies.

In the second week of a sprint, Tuesday from 10:00 to 11:00 the sprint retrospective meeting is held. The team discuss and evaluate the last sprint, and learn from mistakes and successes. The wisdom gained in the retrospective meeting is noted and considered in the Sprint Planning meeting.
\\\\
After the retrospective meeting, Sprint Planning meeting is held from 11:00 to 14:00. In this meeting, user-stories are given a relative time estimate called  story-points. Stories are divided into tasks and chores that must be completed in order to achieve the sprint goal. The outcome is a product backlog with elaborated and estimated tasks. The team can take on any task or chore they see fit or have capacity to do. 
\\\\
A Sprint Log is also made after each Sprint. The Sprint Log describes what have been done during the Sprint, and progress relative to the project plan. This allows the team to learn from each Sprint and to make improvements for the upcoming Sprints. 
\\\\
Scrum reviews are performed after a sprint is ended. The planned days for these meetings are Fridays. The attendees are the team, product owner, scrum master, internal and/or external supervisors. Product increments produced in the last sprint are presented and discussed, any necessary changes and re-prioritization are made.
\\\\
Once a day, the group performs a daily stand-up meeting of no more than 15 minutes. This meeting is referred to as "Daily Scrum" and starts 8:45 AM. 

\clearpage

\section{Management Software, JIRA}
Management-software is used to facilitate Scrum and to track and plan progress. JIRA is the software of choice, containing the backlog, sprint backlog and burndown charts, as well as allocation of human resources.


\section{ShareLaTex}
ShareLaTex is an online LaTeX editing tool where multiple participants can work simultaneously together in the same document. LaTeX, in general, is a high-quality typesetting system; it includes features designed for the production of technical and scientific documentation. LaTeX is the most used tool for the communication and publication of scientific documents \cite{latex}.  


\section{GitHub}
In order to ensure backup of all documents at all times, documents are pushed to GitHub. This guarantees that all documents are saved, in the case of any unforeseen situations that can cause loss of documentation. It also ensures traceability of all document history. 








