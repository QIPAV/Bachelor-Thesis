\textbf{\LARGE Modified Scrum} \\
FFI(Forsvarets Forskningsinstitutt) has requested a scientific research paper on variable pitch quadcopters. To achieve this high level goal, a quadcopter must be designed, built and then tested.\\

Building a quadcopter is an interdisciplinary effort, containing elements from software, hardware and electrical engineering. The project has many variables and few initial requirements. Uncertainty and unknowns are of great concern, and the chances of change during the execution of the project is high. \\

Design, building and testing must all be completed within the project deadline which is less than 5 months. Due to the shear amount of time and work required to accomplish the task at hand, we need to quickly produce product increments to save as much time as possible. Additionally, we have to produce high quality on a relatively low budget.\\

Hence, we wanted a project model that was agile, which could cope with the schedule, uncertainty and handle changes, even late in the project. While still producing the best possible result in the least amount of time. We chose Scrum, in Scrum feedback and close communication is an integrated part of a project- cycle(sprint). The increased amount and frequency of feedback makes adaptations more easily implemented, and increases the chances of customer satisfaction.  \\

Scrum is based on doing just what is necessary to get started and only planning one sprint ahead, and than elaborate the details as work progresses. This works well when deadlines are movable, but since this is a student project, the deadline is fixed. Another concern is that the project scope might drastically change if the customer is unclear on what they want.\\

To solve this issue, we have created a project plan that describes the main activities in the planned sprints. The project plan is represented as a Gantt diagram.  \\

In the agile manifesto and Scrum in general, there is little emphasis on heavy documentation. Documentation is held at a minimum to ensure that as much time as possible is spent developing the actual product. \\

Since this is a student research project and HSN (University Collage Of South-East Norway) has its own requirements regarding documentation. We are going to incorporate more documentation than specified by the Scrum framework. This for example includes extra documentation with traceability of the backlog and tasks. There will also be generated acceptance criteria that has to be met by some form of testing. In addition, much of the project itself consists of documenting tests, advantages and disadvantages of variable pitch quadcopters. \\

For roles and ceremonies we will do as the model specifies, with some minor modifications. The team shall be self- driven and self- organizing and may adapt within their specified roles and tasks as they see fit. \\

We will limit ourselves regarding time-usage by performing time-boxing on all ceremonies. Our "Sprint Review" document will be combined with the follow- up document specified by HSN. This because the two documents will contain about the same information, thus saving time generating two similar documents.\\

In scrum, time estimation is primarily done by the use of story-points. Story-points are relative time estimates, describing the relative size of a task or backlog item in comparison to other tasks or backlog items. Research has shown that estimates done by this method often end up being much more accurate than when we try to make a detailed estimate by adding up the all the pieces. We have chosen to use story-points, and we estimate the sizes of tasks  together as a team by playing estimate poker. Each team member presents a card describing how big they think the task is, we then compare and discuss until a final estimate is approved. 

%Since HSN requires a detailed time- plan, we will not use these story- points and thus we are missing out on some benefits of scrum. Instead story- points are regarded as hours, so that our burn-down chart represents the amount of work done or remaining in hours.  