\section{Radio Controller - Fixed and Variable Pitch Mode with Auto-Level}

During this thesis the team has been working on making a flight controller. Our main objective is to study variable pitch quadrotors and gain valuable information about them. The code presented gives the possibility to fly a quadcopter with variable or fixed pitch on either acro-mode or with auto-level. Beware that the i\_gains increase the drift of the quadcopter, and after long flights the quadcopter might get an angled force vector. 

\begin{lstlisting}
//Include the Servo.h library so we can use servos to adjust pitch on propeller.
#include <Servo.h>      
                   
//Include the Wire.h library so we can communicate with the gyro.
#include <Wire.h>

//Include the EEPROM.h library so we can store information onto the EEPRO                         
#include <EEPROM.h>                        

////////////////////////////////////////////////////////////////////////////////////////
//PID gain and limit settings
////////////////////////////////////////////////////////////////////////////////////////
//Gain setting for the roll P-controller. (11.5 perf for max, 3.5 for critical RPM)
float pid_p_gain_roll = 3.5;              

//Gain setting for the roll I-controller (0.05 perf for max, 0.001 for critical RPM) 
float pid_i_gain_roll = 0.001;             

//Gain setting for the roll D-controller (10 perf for max, 7 for critical RPM)
float pid_d_gain_roll = 7;                

//Maximum output of the PID-controller (+/-)
int pid_max_roll = 400;                    


float pid_p_gain_pitch = pid_p_gain_roll;  //Gain setting for the pitch P-controller.
float pid_i_gain_pitch = pid_i_gain_roll;  //Gain setting for the pitch I-controller.
float pid_d_gain_pitch = pid_d_gain_roll;  //Gain setting for the pitch D-controller.
int pid_max_pitch = pid_max_roll;          //Maximum output of the PID-controller (+/-)
 
float pid_p_gain_yaw = 26.0;        //Gain setting for the pitch P-controller. (18 / 26)
float pid_i_gain_yaw = 0.04;        //Gain setting for the pitch I-controller. (0.04)
float pid_d_gain_yaw = 0.0;         //Gain setting for the pitch D-controller.
int pid_max_yaw = 400;              //Maximum output of the PID-controller (+/-)

boolean auto_level = true;          //Auto level on (true) or off (false)

////////////////////////////////////////////////////////////////////////////////////////
//Declaring global variables
////////////////////////////////////////////////////////////////////////////////////////
byte last_channel_1, last_channel_2, last_channel_3, last_channel_4;
byte eeprom_data[36];
byte highByte, lowByte;
volatile int receiver_input_channel_1, receiver_input_channel_2, 
receiver_input_channel_3, receiver_input_channel_4;
int counter_channel_1, counter_channel_2, counter_channel_3, 
counter_channel_4, loop_counter;
int esc_1, esc_2, esc_3, esc_4, servo_1, servo_2, servo_3, servo_4; //Added Servo
int throttle, battery_voltage;
int cal_int, start, gyro_address;
int receiver_input[5];
int oldest_data = 0;
int number_of_samples = 5;
int temperature;

int acc_axis[4], gyro_axis[4];
float roll_level_adjust, pitch_level_adjust;

long acc_x, acc_y, acc_z, acc_total_vector;
unsigned long timer_channel_1, timer_channel_2, timer_channel_3, timer_channel_4, 
esc_timer, esc_loop_timer;
unsigned long timer_1, timer_2, timer_3, timer_4, current_time;
unsigned long loop_timer;
double gyro_pitch, gyro_roll, gyro_yaw;
double gyro_pitch_sum, gyro_roll_sum, gyro_yaw_sum, acc_pitch_sum, acc_roll_sum;
double gyro_axis_cal[4], acc_axis_cal[3],  gyro_axis_pitch[20], gyro_axis_roll[20], 
gyro_axis_yaw[20], acc_axis_pitch[20], acc_axis_roll[20], acc_axis_yaw[20];
float pid_error_temp;
float pid_i_mem_roll, pid_roll_setpoint, gyro_roll_input, pid_output_roll, 
pid_last_roll_d_error;
float pid_i_mem_pitch, pid_pitch_setpoint, gyro_pitch_input, pid_output_pitch, 
pid_last_pitch_d_error;
float pid_i_mem_yaw, pid_yaw_setpoint, gyro_yaw_input, pid_output_yaw, 
pid_last_yaw_d_error;
float angle_roll_acc, angle_pitch_acc, angle_pitch, angle_roll;
boolean gyro_angles_set;

////////////////////////////////////////////////////////////////////////////////////////
// Calibration values
////////////////////////////////////////////////////////////////////////////////////////
int s1 = 1400;
int s2 = 1575;
int s3 = 1560;
int s4 = 1400;
 
int s_inc = 290;
int s1_max, s2_max, s3_max, s4_max;
bool array_full = false;

int number = 0;
bool debug = false;

//void calibrate_servos();
 
// Servo
Servo servo1;
Servo servo2;
Servo servo3;
Servo servo4;

const float h[] = {0.00399582123524659,  0.00979098916767830,  0.0195392007689703, 
    0.0328394476348149, 0.0486189357392199, 0.0649045550087392, 0.0791796000445175, 
    0.0889684699880066, 0.0924516144752421, 0.0889684699880066, 0.0791796000445175, 
    0.0649045550087392, 0.0486189357392199, 0.0328394476348149, 0.0195392007689703, 
    0.00979098916767830,  0.00399582123524659};

// Defining length of shift register
#define M (sizeof(h)/sizeof(float))  

// Setting all values in shiftregister B as zero value
float x[M]={0};                             
float y[M]={0};
float z[M]={0};
int N = M-1;                                // Filter Order
float angle=0;                              // Output variable for y[n]

////////////////////////////////////////////////////////////////////////////////////////
//Setup routine
////////////////////////////////////////////////////////////////////////////////////////
void setup(){
  DDRD |= B11110000;

  Serial.begin(57600);
  //Copy the EEPROM data for fast access data.
  for(start = 0; start <= 35; start++)eeprom_data[start] = EEPROM.read(start);
  //Set start back to zero.
  start = 0;
  //Store the gyro address in the variable.                                                                
  gyro_address = eeprom_data[32];                                           

  servo1.attach(2); // portE PE4 B00010000
  servo2.attach(3); // portE PE5 B00100000
  servo3.attach(12);// portE PE6 B01000000
  servo4.attach(13);// portE Pe6 B10000000

  //Start the I2C as master.
  Wire.begin();                                                             

  //Set the I2C clock speed to 400kHz.
  TWBR = 12;                                                                

  // Arduino (Atmega) pins default to inputs, 
  // so they don't need to be explicitly declared as inputs.
  DDRD |= B11110000;    //Configure digital poort 4, 5, 6 and 7 as output.

  //Check the EEPROM signature to make sure that the setup program is executed.
  while(eeprom_data[33] != 'J' || eeprom_data[34] != 'M' || 
  eeprom_data[35] != 'B')delay(10);

  //The flight controller needs the MPU-6050 with gyro and accelerometer
  //If setup is completed without MPU-6050 stop the flight controller program  
  if(eeprom_data[31] == 2 || eeprom_data[31] == 3)delay(10);

  set_gyro_registers();       //Set the specific gyro registers.
  //Wait 5 seconds before continuing.
  for (cal_int = 0; cal_int < 1250 ; cal_int ++){                           
    PORTD |= B11110000;       //Set digital poort 4, 5, 6 and 7 high.
    delayMicroseconds(1000);  //Wait 1000us.
    PORTD &= B00001111;       //Set digital poort 4, 5, 6 and 7 low.
    delayMicroseconds(3000);  //Wait 3000us.
  }

  //Let's take multiple gyro data samples
  // so we can determine the average gyro offset (calibration).

  //Take 2000 readings for calibration.
  for (cal_int = 0; cal_int < 2000 ; cal_int ++){
    //Read the gyro output.                          
    gyro_signalen();                                                        

    //Calculate the total accelerometer vector.
    acc_total_vector = sqrt((acc_x*acc_x)+(acc_y*acc_y)+(acc_z*acc_z));     

    //Prevent the asin function to produce a NaN
    if(abs(acc_y) < acc_total_vector){   
      //Calculate the pitch angle.                                   
      angle_pitch_acc = asin((float)acc_y/acc_total_vector)* 57.296;        
    }
    //Prevent the asin function to produce a NaN
    if(abs(acc_x) < acc_total_vector){
      //Calculate the roll angle.                                      
      angle_roll_acc = asin((float)acc_x/acc_total_vector)* -57.296;        
    }
    
    gyro_axis_cal[1] += gyro_axis[1];   //Ad roll value to gyro_roll_cal.
    gyro_axis_cal[2] += gyro_axis[2];   //Ad pitch value to gyro_pitch_cal.
    gyro_axis_cal[3] += gyro_axis[3];   //Ad yaw value to gyro_yaw_cal.
    //Need acc offset too!  
    acc_axis_cal[1] += angle_pitch_acc; //Ad pitch value to acc_roll_cal.
    acc_axis_cal[2] += angle_roll_acc;  //Ad roll value to acc_roll_cal.
    
    //We don't want the esc's to be beeping annoyingly. 
    //So let's give them a 1000us puls while calibrating the gyro.
    PORTD |= B11110000;       //Set digital poort 4, 5, 6 and 7 high.
    delayMicroseconds(1000);  //Wait 1000us.
    PORTD &= B00001111;       //Set digital poort 4, 5, 6 and 7 low.
    delay(3);                 //Wait 3 milliseconds before the next loop.
  }
  // Now that we have 2000 measures.
  // We need to devide by 2000 to get the average gyro offset.
  gyro_axis_cal[1] /= 2000;     //Divide the roll total by 2000.
  gyro_axis_cal[2] /= 2000;     //Divide the pitch total by 2000.
  gyro_axis_cal[3] /= 2000;     //Divide the yaw total by 2000.
  acc_axis_cal[1] /= 2000;      //Divide the pitch_angle sum by 2000.
  acc_axis_cal[2] /= 2000;      //Divide the roll_angle sum by 2000.

  PCICR |= (1 << PCIE0);        //Set PCIE0 to enable PCMSK0 scan.

  //Set PCINT0 (digital input 8) to trigger an interrupt on state change.
  PCMSK0 |= (1 << PCINT0);

  //Set PCINT1 (digital input 9)to trigger an interrupt on state change.
  PCMSK0 |= (1 << PCINT1);
  
  //Set PCINT2 (digital input 10)to trigger an interrupt on state change.                                                 
  PCMSK0 |= (1 << PCINT2); 

  //Set PCINT3 (digital input 11)to trigger an interrupt on state change.
  PCMSK0 |= (1 << PCINT3);                                                  

  //Wait until the receiver is active and the throtle is set to the lower position.
  while(receiver_input_channel_3 < 990 || receiver_input_channel_3 > 1020 
  || receiver_input_channel_4 < 1400){
    
    //Convert the actual receiver signals for throttle to the standard 1000 - 2000us
    receiver_input_channel_3 = convert_receiver_channel(3);
    
    //Convert the actual receiver signals for yaw to the standard 1000 - 2000us                 
    receiver_input_channel_4 = convert_receiver_channel(4);                 
    start ++;   //While waiting increment start whith every loop.
    //We don't want the esc's to be beeping annoyingly. 
    //So let's give them a 1000us puls while waiting for the receiver inputs.
    PORTD |= B11110000;      //Set digital poort 4, 5, 6 and 7 high.
    delayMicroseconds(1000); //Wait 1000us.
    PORTD &= B00001111;      //Set digital poort 4, 5, 6 and 7 low.
    delay(3);                //Wait 3 milliseconds before the next loop.
  }
  start = 0;                 //Set start back to 0.

  //Load the battery voltage to the battery_voltage variable.
  //65 is the voltage compensation for the diode.
  //12.6V equals ~ Analog 0.
  //12.6V equals 1023 analogRead(0).
  //1260 / 1023 = 1.2317.
  //The variable battery_voltage holds 1050 if the battery voltage is 10.5V.
  //battery_voltage = (analogRead(0) + 65) * 1.2317;

  loop_timer = micros();  //Set the timer for the next loop.

  pinMode(2, OUTPUT);
  pinMode(3, OUTPUT);
  pinMode(12, OUTPUT);
  pinMode(13, OUTPUT);

  servo1.writeMicroseconds(s1);
  servo2.writeMicroseconds(s2);
  servo3.writeMicroseconds(s3);
  servo4.writeMicroseconds(s4);
 
  s1_max = s1 - s_inc;
  s2_max = s2 + s_inc;
  s3_max = s3 - s_inc;
  s4_max = s4 + s_inc;
}
////////////////////////////////////////////////////////////////////////////////////////
//Main program loop
////////////////////////////////////////////////////////////////////////////////////////
void loop(){
  //65.5 = 1 deg/sec (check the datasheet of the MPU-6050 for more information).
  
  //Gyro pid input is deg/sec.
  gyro_roll_input = (gyro_roll_input * 0.7) + ((gyro_roll / 65.5) * 0.3);
       
  //Gyro pid input is deg/sec.
  gyro_pitch_input = (gyro_pitch_input * 0.7) + ((gyro_pitch / 65.5) * 0.3);
    
  //Gyro pid input is deg/sec.
  gyro_yaw_input = (gyro_yaw_input * 0.7) + ((gyro_yaw / 65.5) * 0.3);        
  
  //LOW PASS FILTER
  int i = 0;
  for(i=N; i>0; i--) {       // Shifts all previous samples one position
      x[i] = x[i-1];         // Makes room for one new sample
      y[i] = y[i-1];
      z[i] = z[i-1];
  }

  x[0] = acc_x;
  y[0] = acc_y;
  z[0] = acc_z;

  acc_x = h[0] * x[0];       // Compute the convolution x[0]*h[0]7
  acc_y = h[0] * y[0];
  acc_z = h[0] * z[0];
 
  for(i = 1; i <= N; i++) {  // Summing the rest of the products
    acc_x += h[i] * x[i]; // Convolve rest of the inputs with the filter coefficients
    acc_y += h[i] * y[i];
    acc_z += h[i] * z[i];
  }
  
  //AVERAGE FILTER
  //Add to array by removing the oldest data
  gyro_axis_pitch[oldest_data] = gyro_pitch_input;
  gyro_axis_roll[oldest_data] = gyro_roll_input;
  gyro_axis_yaw[oldest_data] = gyro_yaw_input;
  acc_axis_pitch[oldest_data] = acc_x;
  acc_axis_roll[oldest_data] = acc_y;
  acc_axis_yaw[oldest_data] = acc_z;
  
  oldest_data++;
  if( oldest_data >= number_of_samples-1){ // 0, 1, 2 >= 2
    oldest_data = 0;
    array_full = true;
  }
  if(array_full){
    
    //reset variables
    gyro_pitch_sum = 0;
    gyro_roll_sum = 0;
    gyro_yaw_sum = 0;
    acc_pitch_sum = 0;
    acc_roll_sum = 0;
    int acc_yaw_sum = 0;

    //get sum
    for(int i = 0; i < number_of_samples; i++){ //0, 1, 2
      gyro_pitch_sum += gyro_axis_pitch[i];
      gyro_roll_sum  += gyro_axis_roll[i];
      gyro_yaw_sum   += gyro_axis_yaw[i];
      acc_pitch_sum  += acc_axis_pitch[i];
      acc_roll_sum   += acc_axis_roll[i];
      acc_yaw_sum    += acc_axis_yaw[i];
    }
    
    //get average
    gyro_pitch_sum /= number_of_samples;
    gyro_roll_sum /= number_of_samples;
    gyro_yaw_sum /= number_of_samples;
    acc_pitch_sum /= number_of_samples;
    acc_roll_sum /= number_of_samples;
    acc_yaw_sum /= number_of_samples;
    
    gyro_pitch_input = gyro_pitch_sum/5;
    gyro_roll_input = gyro_roll_sum/5;
    gyro_yaw_input = gyro_yaw_sum/5;
    acc_x = acc_pitch_sum;
    acc_y = acc_roll_sum;
    acc_z = acc_yaw_sum;
  }
  
  //Accelerometer angle calculations

  //Calculate the total accelerometer vector
  acc_total_vector = sqrt((acc_x*acc_x)+(acc_y*acc_y)+(acc_z*acc_z));       

  //Prevent the asin function to produce a NaN
  if(abs(acc_y) < acc_total_vector){   
    //Calculate the pitch angle.
    angle_pitch_acc = asin((float)acc_y/acc_total_vector)* 57.296;          
  }
  //Prevent the asin function to produce a NaN
  if(abs(acc_x) < acc_total_vector){
    //Calculate the roll angle.                                       
    angle_roll_acc = asin((float)acc_x/acc_total_vector)* -57.296;          
  }

  //remove offset
  angle_pitch_acc -= acc_axis_cal[1];  //Accelerometer calibration value for pitch.
  angle_roll_acc -= acc_axis_cal[2];   //Accelerometer calibration value for roll.
 
  //////////////////////////////////////////////////////////////////////////////////////
  //IMU code 
  //////////////////////////////////////////////////////////////////////////////////////  

  //Gyro angle calculations
  //0.0000611 = 1 / (250Hz / 65.5)
  //Calculate the traveled pitch angle and add this to the angle_pitch variable.
  angle_pitch += gyro_pitch * 0.0000611;                                    

  //Calculate the traveled roll angle and add this to the angle_roll variable.
  angle_roll += gyro_roll * 0.0000611;                                     

  //0.000001066 = 0.0000611 * (3.142(PI) / 180degr) 
  //The Arduino sin function is in radians
  
  //If the IMU has yawed transfer the roll angle to the pitch angel.
  angle_pitch -= angle_roll * sin(gyro_yaw * 0.000001066);
  //If the IMU has yawed transfer the pitch angle to the roll angel.                  
  angle_roll += angle_pitch * sin(gyro_yaw * 0.000001066);                 
  
  //Place the MPU-6050 spirit level and calibrate the acc offset.
  //Accelerometer calibration value for pitch. 
  angle_pitch_acc -= acc_axis_cal[1];
  //Accelerometer calibration value for roll.                                       
  angle_roll_acc -= acc_axis_cal[2];                                        
  
  if(gyro_angles_set){
    //Correct the drift of the gyro pitch angle with the accelerometer pitch angle.
    angle_pitch = angle_pitch * 0.9996 + angle_pitch_acc * 0.0004;
    //Correct the drift of the gyro roll angle with the accelerometer roll angle.
    angle_roll = angle_roll * 0.9996 + angle_roll_acc * 0.0004;               
  }

  pitch_level_adjust = angle_pitch * 15; //Calculate the pitch angle correction
  roll_level_adjust = angle_roll * 15;   //Calculate the roll angle correction

  if(!auto_level){                   //If the quadcopter is not in auto-level mode
    pitch_level_adjust = 0;          //Set the pitch angle correction to zero.
    roll_level_adjust = 0;           //Set the roll angle correcion to zero.
  }


  //For starting the motors: throttle low and yaw left (step 1).
  if(receiver_input_channel_3 < 1050 
  && receiver_input_channel_4 < 1050)start = 1;
  //When yaw stick is back in the center position start the motors (step 2).
  if(start == 1 && receiver_input_channel_3 < 1050 
  && receiver_input_channel_4 > 1450){
    start = 2;

    //Set the gyro pitch angle equal to the 
    //accelerometer pitch angle when the quadcopter is started.
    angle_pitch = angle_pitch_acc;  

    //Set the gyro roll angle equal to the 
    //accelerometer roll angle when the quadcopter is started.
    angle_roll = angle_roll_acc;                                            
    gyro_angles_set = true;        //Set the IMU started flag.

    //Reset the PID controllers for a bumpless start.
    pid_i_mem_roll = 0;
    pid_last_roll_d_error = 0;
    pid_i_mem_pitch = 0;
    pid_last_pitch_d_error = 0;
    pid_i_mem_yaw = 0;
    pid_last_yaw_d_error = 0;
  }
  //Stopping the motors: throttle low and yaw right.
  if(start == 2 && receiver_input_channel_3 < 1050 
  && receiver_input_channel_4 > 1950)start = 0;

  //The PID set point in degrees per second. 
  //And is determined by the roll receiver input.
  //In the case of deviding by 3 the max roll rate
  //is aprox 164 degrees per second ( (500-8)/3 = 164d/s ).
  pid_roll_setpoint = 0;
  //We need a little dead band of 16us for better results.
  if(receiver_input_channel_1 > 1508)
    pid_roll_setpoint = receiver_input_channel_1 - 1508;
  else if(receiver_input_channel_1 < 1492)
    pid_roll_setpoint = receiver_input_channel_1 - 1492;

  //Subtract the angle correction from the standardized receiver roll input value.
  pid_roll_setpoint -= roll_level_adjust;
  //Divide the setpoint for the PID roll controller by 3 to get angles in degrees.                                   
  pid_roll_setpoint /= 20.0;                                                 
  //////////////////////////////////////////////////////////////////////////////////////  

  //The PID set point in degrees per second is determined by the pitch receiver input.
  //In the case of deviding by 3 the max pitch rate 
  //is aprox 164 degrees per second ( (500-8)/3 = 164d/s ).
  pid_pitch_setpoint = 0;
  //We need a little dead band of 16us for better results.
  if(receiver_input_channel_2 > 1508)
    pid_pitch_setpoint = receiver_input_channel_2 - 1508;
  else if(receiver_input_channel_2 < 1492)
    pid_pitch_setpoint = receiver_input_channel_2 - 1492;

  //Subtract the angle correction from the standardized receiver pitch input value.
  pid_pitch_setpoint -= pitch_level_adjust;  
  //Divide the setpoint for the PID pitch controller by 3 to get angles in degrees.                                 
  pid_pitch_setpoint /= 20.0;                                                 

  //The PID set point in degrees per second is determined by the yaw receiver input.
  //In the case of deviding by 3 the max yaw rate 
  //is aprox 164 degrees per second ( (500-8)/3 = 164d/s ).
  pid_yaw_setpoint = 0;
  //We need a little dead band of 16us for better results.
  if(receiver_input_channel_3 > 1050){ //Do not yaw when turning off the motors.
    if(receiver_input_channel_4 > 1508)
      pid_yaw_setpoint = (receiver_input_channel_4 - 1508)/40.0;
    else if(receiver_input_channel_4 < 1492)
      pid_yaw_setpoint = (receiver_input_channel_4 - 1492)/40.0;
  }
  
  calculate_pid(); //PID inputs are known. So we can calculate the pid output.

  //The battery voltage is needed for compensation.
  //A complementary filter is used to reduce noise.
  //0.09853 = 0.08 * 1.2317.
  //battery_voltage = battery_voltage * 0.92 + (analogRead(0) + 65) * 0.09853;

  //Turn on the led if battery voltage is to low.
  //if(battery_voltage < 1000 && battery_voltage > 600)digitalWrite(12, HIGH);

  throttle = receiver_input_channel_3;  //We need the throttle signal as a base signal.

  if (start == 2){ //The motors are started.
    //We need some room to keep full control at full throttle.
    
    if (throttle > 1800) throttle = 1800; 
    //Calculate the pulse for esc 1 (front-right - CCW)
    esc_1 = (throttle - pid_output_pitch + pid_output_roll - pid_output_yaw)*1.044; 
    //Calculate the pulse for esc 2 (rear-right - CW)
    esc_2 = (throttle + pid_output_pitch + pid_output_roll + pid_output_yaw)*1.046;
    //Calculate the pulse for esc 3 (rear-left - CCW)
    esc_3 = (throttle + pid_output_pitch - pid_output_roll - pid_output_yaw)*0.986; 
    //*1.022; //Calculate the pulse for esc 4 (front-left - CW)
    esc_4 = (throttle - pid_output_pitch - pid_output_roll + pid_output_yaw)*1.034; 
    
    /*
    if (battery_voltage < 1240 && battery_voltage > 800){ //Is the battery connected?
      //Compensate the esc's pulse for voltage drop.
      esc_1 += esc_1 * ((1240 - battery_voltage)/(float)3500);              
      esc_2 += esc_2 * ((1240 - battery_voltage)/(float)3500);              
      esc_3 += esc_3 * ((1240 - battery_voltage)/(float)3500);              
      esc_4 += esc_4 * ((1240 - battery_voltage)/(float)3500);              
    } 
    */
    
    if (esc_1 < 1000) esc_1 = 1000;   //Keep the motors running.
    if (esc_2 < 1000) esc_2 = 1000;   //Keep the motors running.
    if (esc_3 < 1000) esc_3 = 1000;   //Keep the motors running.
    if (esc_4 < 1000) esc_4 = 1000;   //Keep the motors running.

    if(esc_1 > 2000)esc_1 = 2000;     //Limit the esc-1 pulse to 2000us.
    if(esc_2 > 2000)esc_2 = 2000;     //Limit the esc-2 pulse to 2000us.
    if(esc_3 > 2000)esc_3 = 2000;     //Limit the esc-3 pulse to 2000us.
    if(esc_4 > 2000)esc_4 = 2000;     //Limit the esc-4 pulse to 2000us.  


    servo_1 = map(esc_1, 1000, 2000, s1, s1_max); 
    servo_2 = map(esc_2, 1000, 2000, s2, s2_max);
    servo_3 = map(esc_3, 1000, 2000, s3, s3_max);
    servo_4 = map(esc_4, 1000, 2000, s4, s4_max); 

    esc_1 = map(esc_1, 1000, 2000, 1700, 1800);
    esc_2 = map(esc_2, 1000, 2000, 1700, 1800);
    esc_3 = map(esc_3, 1000, 2000, 1700, 1800);
    esc_4 = map(esc_4, 1000, 2000, 1700, 1800);
    
    //send servo val 
    servo1.writeMicroseconds(servo_1);
    servo2.writeMicroseconds(servo_2);
    servo3.writeMicroseconds(servo_3);
    servo4.writeMicroseconds(servo_4);

    if (debug){
      /*
      Serial.print(esc_1);
      Serial.print(" ");
      Serial.print(esc_2);
      Serial.print(" ");
      Serial.print(esc_3);
      Serial.print(" ");
      Serial.print(esc_4);
      
      Serial.print(" ");
      
      Serial.print(pid_output_pitch); //high
      Serial.print(" ");
      Serial.print(pid_output_roll); //low
      Serial.print(" ");
      Serial.println(pid_output_yaw); //nothing
      */
      
      //Serial.print(acc_total_vector);
      /*
      Serial.print(" ");
      Serial.print(acc_x); //high
      Serial.print(" ");
      Serial.print(acc_y); //low
      Serial.print(" ");
      Serial.print(acc_z); //nothing
      */
      
      Serial.print(angle_pitch_acc);
      Serial.print(" ");
      Serial.print(angle_roll_acc);
      
      /*
      Serial.print(acc_axis[1]);
      Serial.print(" ");
      Serial.print(" ");
      Serial.print(acc_axis[3]); //low
      Serial.print(" ");
      Serial.print(gyro_axis[1]); //nothing
      Serial.print(" ");
      Serial.print(gyro_axis[2]);
      Serial.print(" ");
      Serial.println(gyro_axis[3]);
      */
      
      Serial.print(" ");
      Serial.print(gyro_roll_input);
      Serial.print(" ");
      Serial.print(gyro_pitch_input);
      Serial.print(" ");
      Serial.println(gyro_yaw_input);
      
      /*
      Serial.print(pid_output_pitch);
      Serial.print(" = ");
      Serial.print(pid_p_gain_pitch);
      Serial.print(" * ");
      Serial.print(pid_error_temp);
      Serial.print(" + ");
      Serial.print(pid_i_mem_pitch);
      Serial.print(" + ");
      Serial.print(pid_d_gain_pitch);
      Serial.print(" * (");
      Serial.print(pid_error_temp);
      Serial.print(" - ");
      Serial.println(pid_last_pitch_d_error);
      */
      /*
      Serial.print(" ");
      Serial.print(pitch_level_adjust);
      Serial.print(" ");
      Serial.println(roll_level_adjust);
      */
    }
  }

  else{
    esc_1 = 1000; //If start is not 2 keep a 1000us pulse for ess-1.
    esc_2 = 1000; //If start is not 2 keep a 1000us pulse for ess-2.
    esc_3 = 1000; //If start is not 2 keep a 1000us pulse for ess-3.
    esc_4 = 1000; //If start is not 2 keep a 1000us pulse for ess-4.
    servo1.writeMicroseconds(s1);
    servo2.writeMicroseconds(s2);
    servo3.writeMicroseconds(s3);
    servo4.writeMicroseconds(s4);
  }

  //////////////////////////////////////////////////////////////////////////////////////  
  //Creating the pulses for the ESC's
  //////////////////////////////////////////////////////////////////////////////////////  

  //! ! ! ! ! ! ! ! ! ! ! ! ! ! ! ! ! ! ! ! ! ! ! ! ! ! ! ! ! ! ! ! ! ! ! ! ! ! ! ! ! !
  //Because of the angle calculation the loop time is getting very important. If the 
  //loop time is longer or shorter than 4000us the angle calculation is off. If you 
  // modify the code make sure that the loop time is still 4000us and no longer! 
  //! ! ! ! ! ! ! ! ! ! ! ! ! ! ! ! ! ! ! ! ! ! ! ! ! ! ! ! ! ! ! ! ! ! ! ! ! ! ! ! ! !                
  
  //All the information for controlling the motor's is available.
  //The refresh rate is 250Hz. That means the esc's need there pulse every 4ms.

  while(micros() - loop_timer < 4000);   //We wait until 4000us are passed.
  loop_timer = micros();                 //Set the timer for the next loop.

  PORTD |= B11110000;                    //Set digital outputs 4,5,6 and 7 high.
  //Calculate the time of the faling edge of the esc's pulse.
  timer_channel_1 = esc_1 + loop_timer;  
  timer_channel_2 = esc_2 + loop_timer;                                     
  timer_channel_3 = esc_3 + loop_timer;                                     
  timer_channel_4 = esc_4 + loop_timer;                                     
  
  //There is always 1000us of spare time. 
  //So let's do something usefull that is very time consuming.
  //Get the current gyro and receiver data
  // and scale it to degrees per second for the pid calculations.
  gyro_signalen();

  //Stay in this loop until output 4,5,6 and 7 are low.
  while(PORTD >= 16){
    //Read the current time.
    esc_loop_timer = micros();                                              
    //Set digital output 4,5,6 and 7 to low if the time is expired.
    if(timer_channel_1 <= esc_loop_timer)PORTD &= B11101111;                
    if(timer_channel_2 <= esc_loop_timer)PORTD &= B11011111;                
    if(timer_channel_3 <= esc_loop_timer)PORTD &= B10111111;                
    if(timer_channel_4 <= esc_loop_timer)PORTD &= B01111111;                
  }
}

////////////////////////////////////////////////////////////////////////////////////////  
//This routine is called every time input 8, 9, 10 or 11 changed state. 
//This is used to read the receiver signals. 
////////////////////////////////////////////////////////////////////////////////////////  
ISR(PCINT0_vect){
  current_time = micros();
  //Channel 1=========================================
  if(PINB & B00000001){         //Is input 8 high?
    if(last_channel_1 == 0){    //Input 8 changed from 0 to 1.
      last_channel_1 = 1;       //Remember current input state.
      timer_1 = current_time;   //Set timer_1 to current_time.
    }
  }
  else if(last_channel_1 == 1){ //Input 8 is not high and changed from 1 to 0.
    last_channel_1 = 0;         //Remember current input state.
    receiver_input[1] = current_time - timer_1; //Channel 1 is current_time - timer_1.
  }
  //Channel 2=========================================
  if(PINB & B00000010 ){        //Is input 9 high?
    if(last_channel_2 == 0){    //Input 9 changed from 0 to 1.
      last_channel_2 = 1;       //Remember current input state.
      timer_2 = current_time;   //Set timer_2 to current_time.
    }
  }
  else if(last_channel_2 == 1){ //Input 9 is not high and changed from 1 to 0.
    last_channel_2 = 0;         //Remember current input state.
    receiver_input[2] = current_time - timer_2; //Channel 2 is current_time - timer_2.
  }
  //Channel 3=========================================
  if(PINB & B00000100 ){        //Is input 10 high?
    if(last_channel_3 == 0){    //Input 10 changed from 0 to 1.
      last_channel_3 = 1;       //Remember current input state.
      timer_3 = current_time;   //Set timer_3 to current_time.
    }
  }
  else if(last_channel_3 == 1){ //Input 10 is not high and changed from 1 to 0.
    last_channel_3 = 0;         //Remember current input state.
    receiver_input[3] = current_time - timer_3; //Channel 3 is current_time - timer_3.

  }
  //Channel 4=========================================
  if(PINB & B00001000 ){        //Is input 11 high?
    if(last_channel_4 == 0){    //Input 11 changed from 0 to 1.
      last_channel_4 = 1;       //Remember current input state.
      timer_4 = current_time;   //Set timer_4 to current_time.
    }
  }
  else if(last_channel_4 == 1){ //Input 11 is not high and changed from 1 to 0.
    last_channel_4 = 0; //Remember current input state.
    receiver_input[4] = current_time - timer_4; //Channel 4 is current_time - timer_4.
  }
}

////////////////////////////////////////////////////////////////////////////////////////  
//Subroutine for reading the gyro
////////////////////////////////////////////////////////////////////////////////////////  
void gyro_signalen(){
  //Read the MPU-6050
  if(eeprom_data[31] == 1){
    Wire.beginTransmission(gyro_address);                                   
    //Start communication with the gyro.
    Wire.write(0x3B);                                                       
    //Start reading @ register 43h and auto increment with every read.
    Wire.endTransmission();                                                 
    //End the transmission.
    Wire.requestFrom(gyro_address,14);                                      
    //Request 14 bytes from the gyro.
    
    receiver_input_channel_1 = convert_receiver_channel(1);                 
    //Convert the actual receiver signals for pitch to the standard 1000 - 2000us.
    receiver_input_channel_2 = convert_receiver_channel(2);                 
    //Convert the actual receiver signals for roll to the standard 1000 - 2000us.
    receiver_input_channel_3 = convert_receiver_channel(3);                 
    //Convert the actual receiver signals for throttle to the standard 1000 - 2000us.
    receiver_input_channel_4 = convert_receiver_channel(4);                 
    //Convert the actual receiver signals for yaw to the standard 1000 - 2000us.
    
    while(Wire.available() < 14);                                           
    //Wait until the 14 bytes are received.
    acc_axis[1] = Wire.read()<<8|Wire.read();                               
    //Add the low and high byte to the acc_x variable.
    acc_axis[2] = Wire.read()<<8|Wire.read();                               
    //Add the low and high byte to the acc_y variable.
    acc_axis[3] = Wire.read()<<8|Wire.read();                               
    //Add the low and high byte to the acc_z variable.
    temperature = Wire.read()<<8|Wire.read();                               
    //Add the low and high byte to the temperature variable.
    gyro_axis[1] = Wire.read()<<8|Wire.read();                              
    //Read high and low part of the angular data.
    gyro_axis[2] = Wire.read()<<8|Wire.read();                              
    //Read high and low part of the angular data.
    gyro_axis[3] = Wire.read()<<8|Wire.read();                              
    //Read high and low part of the angular data.
  }

  if(cal_int == 2000){
    gyro_axis[1] -= gyro_axis_cal[1];                                       
    //Only compensate after the calibration.
    gyro_axis[2] -= gyro_axis_cal[2];                                       
    //Only compensate after the calibration.
    gyro_axis[3] -= gyro_axis_cal[3];                                       
    //Only compensate after the calibration.
  }
  gyro_roll = gyro_axis[eeprom_data[28] & 0b00000011];                      
  //Set gyro_roll to the correct axis that was stored in the EEPROM.
  if(eeprom_data[28] & 0b10000000)gyro_roll *= -1;                          
  //Invert gyro_roll if the MSB of EEPROM bit 28 is set.
  gyro_pitch = gyro_axis[eeprom_data[29] & 0b00000011];                     
  //Set gyro_pitch to the correct axis that was stored in the EEPROM.
  if(eeprom_data[29] & 0b10000000)gyro_pitch *= -1;                         
  //Invert gyro_pitch if the MSB of EEPROM bit 29 is set.
  gyro_yaw = gyro_axis[eeprom_data[30] & 0b00000011];                       
  //Set gyro_yaw to the correct axis that was stored in the EEPROM.
  if(eeprom_data[30] & 0b10000000)gyro_yaw *= -1;                           
  //Invert gyro_yaw if the MSB of EEPROM bit 30 is set.

  acc_x = acc_axis[eeprom_data[29] & 0b00000011];                           
  //Set acc_x to the correct axis that was stored in the EEPROM.
  if(eeprom_data[29] & 0b10000000)acc_x *= -1;                              
  //Invert acc_x if the MSB of EEPROM bit 29 is set.
  acc_y = acc_axis[eeprom_data[28] & 0b00000011];                           
  //Set acc_y to the correct axis that was stored in the EEPROM.
  if(eeprom_data[28] & 0b10000000)acc_y *= -1;                              
  //Invert acc_y if the MSB of EEPROM bit 28 is set.
  acc_z = acc_axis[eeprom_data[30] & 0b00000011];                           
  //Set acc_z to the correct axis that was stored in the EEPROM.
  if(eeprom_data[30] & 0b10000000)acc_z *= -1;                              
  //Invert acc_z if the MSB of EEPROM bit 30 is set.
}

////////////////////////////////////////////////////////////////////////////////////////  
//Subroutine for calculating pid outputs
////////////////////////////////////////////////////////////////////////////////////////  
void calculate_pid(){
  //Roll calculations
  pid_error_temp = gyro_roll_input - pid_roll_setpoint;
  pid_i_mem_roll += pid_i_gain_roll * pid_error_temp;
  if(pid_i_mem_roll > pid_max_roll)pid_i_mem_roll = pid_max_roll;
  else if(pid_i_mem_roll < pid_max_roll * -1)pid_i_mem_roll = pid_max_roll * -1;

  pid_output_roll = pid_p_gain_roll * pid_error_temp + pid_i_mem_roll 
  + pid_d_gain_roll * (pid_error_temp - pid_last_roll_d_error);
  if(pid_output_roll > pid_max_roll)pid_output_roll = pid_max_roll;
  else if(pid_output_roll < pid_max_roll * -1)pid_output_roll = pid_max_roll * -1;

  pid_last_roll_d_error = pid_error_temp;

  //Pitch calculations
  pid_error_temp = gyro_pitch_input - pid_pitch_setpoint;
  pid_i_mem_pitch += pid_i_gain_pitch * pid_error_temp;
  if(pid_i_mem_pitch > pid_max_pitch)pid_i_mem_pitch = pid_max_pitch;
  else if(pid_i_mem_pitch < pid_max_pitch * -1)pid_i_mem_pitch = pid_max_pitch * -1;

  pid_output_pitch = pid_p_gain_pitch * pid_error_temp + pid_i_mem_pitch 
  + pid_d_gain_pitch * (pid_error_temp - pid_last_pitch_d_error);
  if(pid_output_pitch > pid_max_pitch)pid_output_pitch = pid_max_pitch;
  else if(pid_output_pitch < pid_max_pitch * -1)pid_output_pitch = pid_max_pitch * -1;

  pid_last_pitch_d_error = pid_error_temp;

  //Yaw calculations
  pid_error_temp = gyro_yaw_input - pid_yaw_setpoint;
  pid_i_mem_yaw += pid_i_gain_yaw * pid_error_temp;
  if(pid_i_mem_yaw > pid_max_yaw)pid_i_mem_yaw = pid_max_yaw;
  else if(pid_i_mem_yaw < pid_max_yaw * -1)pid_i_mem_yaw = pid_max_yaw * -1;

  pid_output_yaw = pid_p_gain_yaw * pid_error_temp + pid_i_mem_yaw + pid_d_gain_yaw 
  * (pid_error_temp - pid_last_yaw_d_error);
  if(pid_output_yaw > pid_max_yaw)pid_output_yaw = pid_max_yaw;
  else if(pid_output_yaw < pid_max_yaw * -1)pid_output_yaw = pid_max_yaw * -1;

  pid_last_yaw_d_error = pid_error_temp;
}

//The stored data in the EEPROM is used.
int convert_receiver_channel(byte function){
  byte channel, reverse;                                                       
  //First we declare some local variables
  int low, center, high, actual;
  int difference;

  channel = eeprom_data[function + 23] & 0b00000111;                           
  //What channel corresponds with the specific function
  if(eeprom_data[function + 23] & 0b10000000)reverse = 1;                      
  //Reverse channel when most significant bit is set
  else reverse = 0;                                                            
  //If the most significant is not set there is no reverse

  actual = receiver_input[channel];                                            
  //Read the actual receiver value for the corresponding function
  low = (eeprom_data[channel * 2 + 15] << 8) | eeprom_data[channel * 2 + 14];  
  //Store the low value for the specific receiver input channel
  center = (eeprom_data[channel * 2 - 1] << 8) | eeprom_data[channel * 2 - 2]; 
  //Store the center value for the specific receiver input channel
  high = (eeprom_data[channel * 2 + 7] << 8) | eeprom_data[channel * 2 + 6];   
  //Store the high value for the specific receiver input channel

  if(actual < center){                                                         
    //The actual receiver value is lower than the center value
    if(actual < low)actual = low;                                              
    //Limit the lowest value to the value that was detected during setup
    difference = ((long)(center - actual) * (long)500) / (center - low);       
    //Calculate and scale the actual value to a 1000 - 2000us value
    if(reverse == 1)return 1500 + difference;                                  
    //If the channel is reversed
    else return 1500 - difference;                                             
    //If the channel is not reversed
  }
  else if(actual > center){                                                                        
    //The actual receiver value is higher than the center value
    if(actual > high)actual = high;                                            
    //Limit the lowest value to the value that was detected during setup
    difference = ((long)(actual - center) * (long)500) / (high - center);      
    //Calculate and scale the actual value to a 1000 - 2000us value
    if(reverse == 1)return 1500 - difference;                                  
    //If the channel is reversed
    else return 1500 + difference;                                             
    //If the channel is not reversed
  }
  else return 1500;
}

void set_gyro_registers(){
  //Setup the MPU-6050
  if(eeprom_data[31] == 1){
    Wire.beginTransmission(gyro_address);                                      
    //Start communication with the address found during search.
    Wire.write(0x6B);                                                          
    //We want to write to the PWR_MGMT_1 register (6B hex)
    Wire.write(0x00);                                                          
    //Set the register bits as 00000000 to activate the gyro
    Wire.endTransmission();                                                    
    //End the transmission with the gyro.

    Wire.beginTransmission(gyro_address);                                      
    //Start communication with the address found during search.
    Wire.write(0x1B);                                                          
    //We want to write to the GYRO_CONFIG register (1B hex)
    Wire.write(0x08);                                                          
    //Set the register bits as 00001000 (500dps full scale)
    Wire.endTransmission();                                                    
    //End the transmission with the gyro

    Wire.beginTransmission(gyro_address);                                      
    //Start communication with the address found during search.
    Wire.write(0x1C);                                                          
    //We want to write to the ACCEL_CONFIG register (1A hex)
    Wire.write(0x10);                                                          
    //Set the register bits as 00010000 (+/- 8g full scale range)
    Wire.endTransmission();                                                    
    //End the transmission with the gyro

    //Let's perform a random register check to see if the values are written correct
    Wire.beginTransmission(gyro_address);                                      
    //Start communication with the address found during search
    Wire.write(0x1B);                                                          
    //Start reading @ register 0x1B
    Wire.endTransmission();                                                    
    //End the transmission
    Wire.requestFrom(gyro_address, 1);                                         
    //Request 1 bytes from the gyro
    while(Wire.available() < 1);                                               
    //Wait until the 6 bytes are received
    if(Wire.read() != 0x08){                                                   
      //Check if the value is 0x08
      while(1)delay(10);                                                       
      //Stay in this loop for ever
    }

    Wire.beginTransmission(gyro_address);                                      
    //Start communication with the address found during search
    Wire.write(0x1A);                                                          
    //We want to write to the CONFIG register (1A hex)
    Wire.write(0x03);                                                          
    //Set the register bits as 00000011 (Set Digital Low Pass Filter to ~43Hz)
    Wire.endTransmission();                                                    
    //End the transmission with the gyro    

  }  
}
\end{lstlisting}