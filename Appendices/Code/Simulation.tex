\lstset { %
    language=Python,
    backgroundcolor=\color{black!5}, % set backgroundcolor
    basicstyle=\footnotesize,% basic font setting
}


\section{Quadcopter Simulation - Python}

Description and comment of the code: \\
\\
The simulation was created in January as a preliminary study of quadrotors 
and how they compute. 

\begin{lstlisting}
from pylab import *
from mpl_toolkits.mplot3d import Axes3D
import time
 
class QuadCopter:
    def __init__(self):
        # Initalize properties of the quadcopter
        # object 0 is COM, 1-4 is the propellers
        self.pos = zeros((5,3))     # position of all objects
        self.vel = zeros(3)         # velocity of COM
        
        # angles of all objects, in order to allow variable pitch easier
        self.angle = zeros((5,3))   
        self.angle_vel = zeros(3)   # angular velocity of all objects
        
        # mass of each object, change to a custom array when you want to
        self.mass = ones(5)         
        self.initialize_thrust_vector()
 
        # Initialize propeller position
        self.length_arm = 1.0
        self.calculate_propeller_position() # to initialize propeller positions
 
    def initialize_thrust_vector(self):
        '''
        Initializes/resets thrust_vector
        '''
        self.thrust_vec = array([[0.,0.,0.],[0.,0.,1.],[0.,0.,1.],[0.,0.,1.],[0.,0.,1.]])
 
    def calculate_propeller_position(self):
        '''
        Calculates and updates the position of the propellers.
        origo : determines if the position is based on origo or COM position.
                use True when doing force manipulation to simplify torque, False when plotting
        '''
        for i in range(1,5):
            meow = 0.5*pi*(i-0.5)
            self.pos[i] = array([self.length_arm * sin(meow), self.length_arm * cos(meow), 0])

        self.initialize_thrust_vector()
        # print self.pos[1]-self.pos[0], self.thrust_vec[1]
        # print 'initdot ', dot(self.pos[1]-self.pos[0], self.thrust_vec[1])
        self.rotate()
 
 
    def rotate(self):
        '''
        Rotates the position array based on Euler angles psi, theta, phi
            for variable pitch, thrust_vec needs to be rotated with the angle 
            the propeller is rotates as well
            basically just another rotation.
        '''
        phi = self.angle[0,0]
        theta = self.angle[0,1]
        psi = self.angle[0,2]
        # For fixed pitch
        
        
        A = matrix([[cos(theta)*cos(phi),cos(theta)*sin(phi),-sin(theta)],
                   [sin(psi)*sin(theta)*cos(phi)-cos(psi)*sin(phi),sin(psi)
                   *sin(phi)*sin(theta)+cos(psi)*cos(phi),cos(theta)*sin(psi)]
                   ,[cos(psi)*sin(theta)*cos(phi)+sin(psi)*sin(phi),cos(psi)
                   *sin(theta)*sin(phi)-sin(psi)*cos(phi),cos(theta)*cos(psi)]])
                   
        # For variable pitch
        A = matrix([[cos(psi)*cos(phi)-cos(theta)*sin(phi)*sin(psi),cos(psi)
                    *sin(phi)+cos(theta)*cos(phi)*sin(psi),sin(psi)*sin(theta)],
                    [-sin(psi)*cos(phi)-cos(theta)*sin(phi)*cos(psi),-sin(psi)
                    *sin(phi)+cos(theta)*cos(phi)*cos(psi),cos(psi)*sin(theta)],
                    [sin(theta)*sin(phi),-sin(theta)*cos(phi),cos(theta)]])
   
        # print self.pos
        for i in range(1,5):
            self.pos[i] = A.dot(self.pos[i])
            self.thrust_vec[i] = A.dot(self.thrust_vec[i])
        # print self.pos
   
        # rotating thrust vector for variable pitch. Code below should work.
        
        '''
        for i in range(1,5):
            psi = self.angle[i,0]
            theta = self.angle[i,1]
            phi = self.angle[i,2]

            A = matrix([[cos(theta)*cos(phi),cos(theta)*sin(phi),-sin(theta)],  
                   [sin(psi)*sin(theta)*cos(phi)-cos(psi)*sin(phi),sin(psi)*sin(phi)
                   *sin(theta)+cos(psi)*cos(phi), cos(theta)*sin(psi)],
                   [cos(psi)*sin(theta)*cos(phi)+sin(psi)*sin(phi),cos(psi)
                   *sin(theta)*sin(phi)-sin(psi)*cos(phi),cos(theta)*cos(psi)]])

            A = matrix([[cos(psi)*cos(phi)-cos(theta)*sin(phi)*sin(psi),cos(psi)
                    *sin(phi)+cos(theta)*cos(phi)*sin(psi),sin(psi)*sin(theta)],
                    [-sin(psi)*cos(phi)-cos(theta)*sin(phi)*cos(psi),-sin(psi)
                    *sin(phi)+cos(theta)*cos(phi)*cos(psi),cos(psi)*sin(theta)],
                    [sin(theta)*sin(phi),-sin(theta)*cos(phi),cos(theta)]])
            
            self.thrust_vec[i] = A.dot(self.thrust_vec[i])
        '''
   
    def get_thrust(self): #generate required thrust to get to the right position
        # Limits
        minThrust = 0
        maxThrust = 25
 
        wPosition = array([0,0,0.5])
        cPosition = self.pos[0]
        #print cPosition
        thrust_x = (wPosition[0] - cPosition[0])/4
        thrust_y = (wPosition[1] - cPosition[1])/4
        thrust_z = (wPosition[2] - cPosition[2])/4
 
        # Need to apply control theory to the amount of thrust here!
       
        # NB, function should only return the total thrust from each propeller, 
        # not a vector.
        return array([[0.,0.,0.], [0.0, 0.0,12.2625],
                                    [0.0, 0.0, 12.2625],
                                    [0.0, 0.0, 12.2625+2],
                                    [0.0, 0.0, 12.2625+2]])
 
    def calculate_forces(self):
        # numpy arrays are fucking awesome
        forces = zeros((5,3))
   
        #gravity
        gravity = array([0, 0, -9.81])
        propeller_thrust = self.get_thrust()    # fetch thrust from each propeller
        # ^ should not be a 5x3 vector, but rather a 5x1 vector
       
        for i in range(5):
            forces[i] += gravity*self.mass[i] + sum(propeller_thrust[i])*self.thrust_vec[i] 
            # change sum(prop_thrust)
            # print 'force ', sum(forces,axis=0)

        return forces
   
   
    def calculate_torque(self,force):
        torque = zeros((5,3))
        torque[1:] = cross(self.pos[1:],force[1:])
	    print self.pos[1:]
	    print torque
	    print force[1]
	    print 'dot ', dot(self.pos[1],force[1])
 
    return torque
 
 
    def velocity_verlet(self,dt):
        inertia = 1.0   # INERTIA NEEDS TO BE IMPLEMENTED 
        (SolidWorks provides some information about it)
       
        # forward euler
        force = self.calculate_forces()
        torque = self.calculate_torque(force)
        self.vel += dt*sum(force,axis=0)/sum(self.mass)
        self.pos[0] += self.vel*dt
        self.angle_vel += dt*sum(torque,axis=0)/inertia
        self.angle[0] += self.angle_vel*dt
        
	    print self.angle[0]
        self.calculate_propeller_position()
        '''
 
        force = self.calculate_forces() # total force at step i
        torque = self.calculate_torque(force)
        self.vel += dt/2. * sum(force,axis=0)/sum(self.mass) 
        # Calculate velocity at half step
        self.pos[0] += self.vel*dt
   
        # Calculate new angle
        # print self.angle_vel
        self.angle_vel += dt/2. * sum(torque,axis=0)/inertia
        # print torque
        # print self.angle_vel

        # print self.angle[0]
        self.angle[0] += self.angle_vel*dt
        # print self.angle[0]

        # Rotate to get correct forces
        self.calculate_propeller_position()
        # print self.pos
   
        force = self.calculate_forces() #calculate forces again with new position
        torque = self.calculate_torque(force) # calculate torque at new position
        # print torque
        self.vel += dt/2. * sum(force,axis=0)/sum(self.mass)
        # print self.angle_vel
        self.angle_vel += dt/2.*sum(torque,axis=0)/inertia
        # print self.angle_vel
	'''	
	raw_input()
 
 
if __name__=='__main__':
   
    meowcopter = QuadCopter()
   
    time_start = 0
    time_end = 10
    N = 1000
    t = linspace(time_start, time_end, N)
    dt = t[1]-t[0]
   
    NumPlots = 100
    iteration = int(round(float(N)/NumPlots)) if(N > NumPlots) else 1
    # print iteration
    ion()
    # fig3d = figure()
    # ax3d = fig3d.add_subplot(111,projection='3d')
   
    # Integrate with velocity verlet
    for i in range(len(t)):
   
        meowcopter.velocity_verlet(dt)
   
        if (i\%iteration == 0):
       	    '''
	    pos = meowcopter.pos
            ax3d.cla()
            ax3d.plot([pos[0,0], pos[0,0]], [pos[0,1],pos[0,1]], [pos[0,2],pos[0,2]],'o')
            ax3d.plot([pos[1,0],pos[1,0]], [pos[1,1], pos[1,1]], [pos[1,2],pos[1,2]],'o')
            ax3d.plot([pos[2,0],pos[2,0]], [pos[2,1], pos[2,1]], [pos[2,2],pos[2,2]],'o')
            ax3d.plot([pos[3,0],pos[3,0]], [pos[3,1], pos[3,1]], [pos[3,2],pos[3,2]],'o')
            ax3d.plot([pos[4,0],pos[4,0]], [pos[4,1], pos[4,1]], [pos[4,2],pos[4,2]],'o')
            ax3d.plot([pos[1,0],pos[3,0]], [pos[1,1], pos[3,1]], [pos[1,2],pos[3,2]])
            ax3d.plot([pos[2,0],pos[4,0]], [pos[2,1], pos[4,1]], [pos[2,2],pos[4,2]])
            ax3d.set_xlim([-2,2])
            ax3d.set_ylim([-2,2])
            ax3d.set_zlim([-2,2])
            draw()'''
	    # fig3d.savefig('meow_%5d.png'%i)
        # raw_input()
    # Creating gif
    
    '''
    import subprocess, os, glob
    subprocess.call('convert -delay 8 -loop 0 meow_*.png meow.gif', shell=True)
    for f in glob.glob('meow_*.png'):
        os.remove(f)
    print 'Meeshon compleet'
    '''
    
    ioff()
    show()

\end{lstlisting}