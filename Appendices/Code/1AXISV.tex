\section{One Axis - Variable Pitch Code}

Description and comment of the code: \\
\\
The one axis variable pitch code is still in testing. It works like the one axis code, but it has more sophisticated methods. It controls at the current time only the servo output, and we plan to add RPM control for the propellers. The next test is to check if it changes the pitch of the propeller correctly and how this effects the rest of the system. 

\begin{lstlisting}
//#include <I2Cdev.h>
#include <Servo.h>
#include <SoftwareSerial.h>
#include <Wire.h>
// IMU stuff start
#include "I2Cdev.h"
#include "MPU9250.h"
MPU9250 accelgyro;
I2Cdev   I2C_M;
int16_t ax, ay, az;
int16_t gx, gy, gz;
int16_t mx, my, mz;
float Axyz[3];
float Gxyz[3];
float pitchDeg;
float rollDeg;
// IMU Stuff end

// Settings
float kp1 = 0.6;    // Set P-term to tune desired angular velocity
float kp2 = 0.8;    // Set P-term to tune difference between m1 and m2
float Kpitch = 0.5;
int thrust = 1390;  // Set thrust
int maxVal = 1590;  // Max thrust
int minVal = 1190;  // Min thrust
int X_oV = 0;       // Set desired angle for ROLL
int Y_oV = 0;       // Set desired angle for PITCH
// Sample Settings
const int num_samples_cal = 2000; // Number of samples in calibration
const int num_samples_cycle = 20; // Number of samples per cycle
const int num_samples_print = 40;
// Set a limit for max deg/sec gyro meassurement
const int maxGyroVal = 50;
const int minGyroVal = -50;
// Sort settings
bool sort = true; // Do we want to remove x lowest and x highest values?
int remove_num_spikes = 5;
 
SoftwareSerial bd(10 , 11); //RX, TX
// Servo
Servo servo1;
Servo servo3;
Servo motor1;
Servo motor3;
 
// Define Variables
int m1; // motor 1-3 = ROLL
int m2;
int m3; // motor 2-4 = PITCH
int m4;
int s1;
int s3;
int counter = 0;
int ledPin = 13;
int servo1ZeroVal = 90;
int servo2ZeroVal = 90;
 
bool first = true; // Is this the first time looping?
bool calibrated = false; // Are sensored calibrated?
// Variables for ROLL
int X_eV;             // Angle error (for roll)
float X_gV;           // For storing accelerometer data
float X_gVArray [num_samples_cycle];  // Put accelerometer data in array
float X_ACal;         // Accelerometer data offset calibration
float X_oVH;          // Desiered angular velocity
float X_gVH;          // For storing Gyro data
float X_gVHArray [num_samples_cycle]; // Put gyro data in array
float X_GCal;         // Gyro data offset calibration
float X_eVH;          // Angualar velocity error
float X_dKraft;       // For storing thrust difference between propellers
float X_gVHmed;       // Average angular velcoity
float X_gVmed;        // Average angle
float X_dPitch;       
 
void setup()
{
  bd.begin(9600);
  bd.setTimeout(50);
  Serial.begin(38400);
  // IMU Stuff start
  Wire.begin();
  accelgyro.initialize();
  // IMU Stuff end
  motor1.attach(6); // ESC pin
  motor3.attach(3); // ESC pin
  servo1.attach(5);
  servo3.attach(9);
  //Leds
  pinMode(ledPin, OUTPUT);
}
 
void loop() {
  // Calibrate
  if (!calibrated) {
    motor1.writeMicroseconds(0);
    motor3.writeMicroseconds(0);
    Serial.println("Break");
    delay(5000);
    Serial.println("Calibrating...");
    while (!calibrated) {
      // IMU Stuff start
      getAccelGyro_Data();
      // IMU Stuff end
      // Read data from accelerometer
      X_gV = rollDeg; // roll
      // Save all accelerometer data in a array
      X_ACal = X_ACal + X_gV;
      // Read data from gyro
      X_gVH = Gxyz[1];   // Acc x
      // Save all gyro data in a array
      X_GCal = X_GCal + X_gVH;
 
      counter++;
 
      if (counter >= num_samples_cal) {
        //calcualte average
        X_ACal = X_ACal / num_samples_cal;
        X_GCal = X_GCal / num_samples_cal;
 
        Serial.println("Finished Calibrating");
        calibrated = true;
        //TURN ON LED
        digitalWrite(ledPin, HIGH); //It's calibrated
 
        motor1.writeMicroseconds(1000);
        motor3.writeMicroseconds(1000);
        delay(3000);
      }
    }
  }
  //ARM MOTORS
  if (first) {
    if (bd.available()) {
      char start = bd.read();
      if (start = '1') { //add start sign
        motor1.writeMicroseconds(1000);
        motor3.writeMicroseconds(1000);
        delay(3000);
        Serial.println("Armed");
        first = false;
      }
    }
  }
  while (!first and calibrated) {
    if (bd.available()) {
      char stopp = bd.read();
      if (stopp = '!') {
        motor1.writeMicroseconds(0);
        motor3.writeMicroseconds(0);
        first = true;
      }
    }
    // IMU STUFF START
    getAccelGyro_Data();
    float   rollDegFiltered = rollDegFiltered * 0.9813 +  0.01867 * rollDeg;  // First order Lowpass filter, Fc at 20 Hz.
    //Read data from accelerometer and gyro
    float x_gVprev = X_gV;
    X_gV = rollDegFiltered - X_ACal;   // Roll - offset
    X_gV = 0.98 * X_gV + (0.02) * x_gVprev; // exponentially weighed moving average filter. Try and fail with constant ,current_output  = α*current_input + (1-α)*previous_output

    //Save all accelerometer data in a array
    X_gVArray[counter % num_samples_cycle] = X_gV;
 
    X_gVH = Gxyz[1];   // Acc x
 
    //limit extreme speed
    if (X_gVH > maxGyroVal) {
      X_gVH = maxGyroVal;
    }
    else if (X_gVH < minGyroVal) {
      X_gVH = minGyroVal;
    }
 
    //Save all gyro data in a different array
    X_gVHArray[counter % num_samples_cycle] = X_gVH - X_GCal; //Measured X - offset
 
    counter++;
 
    if (counter % num_samples_cycle == 0) {
      //Compute for every x readings
 
      //Sort arrays - this might take too much time
      if (sort) {
        int temp;
        for (int i = 0; i < num_samples_cycle - 1; i++) {
          for (int j = 0; j < num_samples_cycle - 1; j++) {
            //Swapping element in if statement
            if (X_gVArray[j] > X_gVArray[j + 1]) {
              temp = X_gVArray[j];
              X_gVArray[j] = X_gVArray[j + 1];
              X_gVArray[j + 1] = temp;
            }
            if (X_gVHArray[j] > X_gVHArray[j + 1]) {
              temp = X_gVHArray[j];
              X_gVHArray[j] = X_gVHArray[j + 1];
              X_gVHArray[j + 1] = temp;
            }
          }
        }
 
        //Add remaining values
        for (int i = remove_num_spikes; i < (num_samples_cycle - remove_num_spikes); i++) {
          X_gVmed = X_gVmed + X_gVArray[i];
          //Find average angular velocity
          X_gVHmed = X_gVHmed + X_gVHArray[i];
        }
 
        //Find average angle
        X_gVmed = X_gVmed / (num_samples_cycle - remove_num_spikes * 2);
        //Find average angular velocity
        X_gVHmed = X_gVHmed / (num_samples_cycle - remove_num_spikes * 2);
      }
      else {
        for (int i = 0; i < num_samples_cycle; i++) {
          X_gVmed = X_gVmed + X_gVArray[i];
          //Find average angular velocity
          X_gVHmed = X_gVHmed + X_gVHArray[i];
        }
 
        //Find average angle
        X_gVmed = X_gVmed / num_samples_cycle;
        //Find average angular velocity
        X_gVHmed = X_gVHmed / num_samples_cycle;
      }
 
      //Error angle
      X_eV = X_oV - X_gVmed;
      //Set desired angular velocity
      X_oVH = X_eV * kp1;
 
      X_eVH = X_oVH - X_gVHmed;
 
      //Thrust differential
      X_dKraft = X_eVH * kp2;
      X_dPitch = X_dKraft * Kpitch;
      //Final thrust
      m1 = thrust + X_dKraft;
      m3 = thrust - X_dKraft;
      s1 = servo1ZeroVal + X_dPitch;
      s3 = servo3ZeroVal - X_dPitch;

 
      //Check if m1 exceeds the limit
      if (m1 > maxVal) {
        m1 = maxVal;
      }
      else if (m1 < minVal) {
        m1 = minVal;
      }
      //Check if m2 exceeds the limit
      if (m2 > maxVal) {
        m2 = maxVal;
      }
      else if (m2 < minVal) {
        m2 = minVal;
      }
      //Check if m3 exceeds the limit
      if (m3 > maxVal) {
        m3 = maxVal;
      }
      else if (m3 < minVal) {
        m3 = minVal;
      }
      //Check if m4 exceeds the limit
      if (m4 > maxVal) {
        m4 = maxVal;
      }
      else if (m4 < minVal) {
        m4 = minVal;
      }
 
      //Send PWM signals
      //motor1.writeMicroseconds(m1);
      //motor3.writeMicroseconds(m3);
      servo1.write(s1);
      servo3.write(s3);
 
    }
    if (counter % num_samples_print == 0) {
      //Print pwm signals for motors, acc and gyro data for every 10 loop
      Serial.print("X : ");
      Serial.print(m1);
      Serial.print(" - ");
      Serial.print(m3);
      Serial.print(" - ");
      Serial.print(X_gVmed);
      Serial.print(" - ");
      Serial.print(X_gVHmed);
      Serial.print(" ");
      Serial.print(X_gVHmed);
      Serial.print(" ");
      Serial.print(s1);
      Serial.print(" ");
      Serial.print(s3);
      Serial.println(" ");
    }
  }
}
 
// IMU FUNCTIONS
void getAccelGyro_Data(void)
{
  accelgyro.getMotion6(&ax, &ay, &az, &gx, &gy, &gz);
  // Accel
  Axyz[0] = ((double) ax / 256) -  6;
  Axyz[1] = ((double) ay / 256) - 4;
  Axyz[2] = ((double) az / 256);
 
  // Gyro
  Gxyz[0] = (double) gx * 250 / 16384;
  Gxyz[1] = (double) gy * 250 / 16384;
  Gxyz[2] = (double) gz * 250 / 32768;
 
  pitchDeg = 180 * (atan(Axyz[0] / sqrt(Axyz[1] * Axyz[1] + Axyz[2] * Axyz[2]))) / PI; // degrees
  rollDeg =  180 * (atan(Axyz[1] / sqrt(Axyz[0] * Axyz[0] + Axyz[2] * Axyz[2]))) / PI;  // degrees
}
\end{lstlisting}