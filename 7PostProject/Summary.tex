\chapter{Project Summary}

The basis of this project is the hypothesis that addition of variable pitch helps the traditional quadcopter overcome its limitations. Traditional fixed pitch quadcopters are limited by the speed at which the propellers can change their rotational speed. The rate of change depends on the inertia of the rotating parts and the torque-speed characteristics of the motors. The responsiveness of fixed pitch quadcopters has shown to be inadequate in turbulent and challenging conditions where fast changes of thrust are required. \bigskip

The theory is that variable pitch gives the possibility of near instant thrust changes. Implementation of variable pitch increases the weight and mechanical complexity of the quadcopter over regular fixed pitch quadcopters. The question is, does the benefits of variable pitch outweigh the changes to the simple and robust fixed pitch quadcopter. \bigskip

Research and development of a variable pitch quadcopter has been carried out on the request from FFI. The objectives were to identify the benefits of using variable pitch over traditional fixed pitch quadcopters. The objectives are to investigate how variable pitch affects the responsiveness of the vehicles, how variable pitch improves flight stability, landings and how it handles ground effect. \bigskip

Initially, two quadcopters were supposed to be built in order to be compared to one another. The first fixed pitch quadcopter built, did never work as intended and was discarded. However, the variable pitch quadcopter was built and did work. The comparison between variable and fixed pitch quadcopters was instead done by fixating the pitch angle of the variable pitch quadcopter, testing it in fixed pitch mode. By using the same quadcopter for comparison, all other factors are kept constant except the addition of variable pitch.\bigskip

In the data found, there were little evidence to support that variable pitch in itself gives more stable flight or landing. The control system and stability never got as optimized as first intended and a full conclusion cannot be drawn based on our findings. The most evident benefit of variable pitch according to the tests, is increased responsiveness. The increased responsiveness is due to the ability of the variable pitch quadcopter to release the kinetic energy stored in the propellers rotation.\bigskip

A carbon quadcopter was designed and built from scratch. For the flight controller, a micro-controller and an IMU was used.



\begin{comment}
- Jan Dyre comments, backup:
- Parameter verification, PID in to VPQ
- Write about our experiences
- Good with 131 tests, nobody expects more
- Auto-level tests - Brutal flight controller
- Accerlerometer tests
- Variable Pitch no RPM compensation
- Spenning / vinkel på x-aksen på simuleringene (spesifikt for fixed pitch)
- Z-axis must be defined as up in VPQ
- More consistent with the word Test (to verifciation). Test -> Research in VPQ.
- Reality vs. project plan
- Forside på høyre side!
- Teknisk tabell feil på mek & mathematical model
- Masse antagelser, på variabel pitch quad, vekt-distrubisjon.
- 5 settninger i auto-level
- 2 settninger på presentasjon
\end{comment}