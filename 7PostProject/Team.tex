\chapter{Team}
%This section provides information about who the project team consisted of.  This usually includes names, titles, project role, and contact information.  This information is useful when questions may arise on future projects which are similar in nature.  It also provides a useful list of points of contact should more information be needed on lessons learned from the project.

Team VAPIQ is a highly motivated interdisciplinary team consisting of six members. The team has an adverse spectrum of knowledge and personal skills. All traditional engineering disciplines are represented, mechanical, electrical and software engineering. For the project all team members have sacrificed their personal time and shown efforts far beyond what was expected. \bigskip

Bringing the people and width of knowledge together has created a well functioning environment where innovation and learning has paved the way forward. All individuals in the team have contributed in their own unique way to successfully build, integrate and test the quadcopters. \bigskip

Below you can find information about the team members. 

\begin{table}[H]
\begin{center}
\textbf{\Large Team Members}\\
\begin{tabular}{lll}
\rowcolor{cadetgrey}
\textbf{Name:}    &\textbf{Discipline:} 	 &\textbf{Project Role:}     \\
                           Tomas Strøm Lyngroth & Mechanical Engineering  & Project Owner/Leader \\ 
\rowcolor{gainsboro}       Stian Fredriksen & Software Engineering  & Software Developer \& Scrum Master \\ %Scrum Master & Software Developer
                           Kent Kjeldaas & Software Engineering & Main Software Developer \\ 
\rowcolor{gainsboro}       Vanja Katinka Halvorsen & Electrical Engineering  & Test Manager \\
                           Katrine Sundal Haune & Electrical Engineering & Documentation Manager \\ 
\rowcolor{gainsboro}       Aleksander Holthe & Electrical Engineering  & Interface Manager \\


\end{tabular} 
\end{center}
\end{table}

The VAPIQ team may be contacted on \textbf{\href{mailto:vapiq@outlook.com}{vapiq@outlook.com}} if more information is needed.  \bigskip

\section{Teamwork}
%\subsection{Work Methodology}

The project was carried out following the agile Scrum framework. The agility of the framework has given transparency within the team and good commutation between all parties and stakeholders. Great communication has helped the team identify and handle changes early. \bigskip

There has been some challenges getting work-flow to function properly between all the disciplines because of the complexity, interdependent tasks and lack of earlier experience. But besides this, the collaboration has been impeccable and the whole team is satisfied with the results achieved in this project. \bigskip

To make the team function, ensure quality in the work, progress and traceability, multiple tools have been utilized. 
\section{Management Software, JIRA}
JIRA project management tool has been used to facilitate Scrum and keep track of all activities carried out during the project.  It has been of great importance for the planning, tracking and distribution of work.

\section{ShareLaTex}
ShareLaTeX is an online LaTeX editor where multiple participant can work simultaneously. It has given the possibility of simultaneously working in one document and eliminated the need for tedious hours of merging documents. It has helped to save time, and big changes can easily be done in a large document with little effort. ShareLaTeX has helped in creating a good document standard and structure, as well as back-up of all documents. LaTeX, in general, is good for writing technical and scientific documentation. 

\section{GitHub}
In order to ensure backup of all documents at all times, everything was pushed to GitHub. This guarantees that all documents are saved, in the case of any unforeseen situations that can cause loss of documentation. It has also ensured traceability in all document history. GitHub is used both for the LaTeX-files and all the code produced in the project.




