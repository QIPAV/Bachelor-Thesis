\chapter*{Project Description}
\Hide 
\part{The Project}
\label{sec:project}
The Norwegian Defence Research Establishment (FFI) has requested a research paper examining the properties of quadcopters with variable pitch propellers.\bigskip

A quadcopter with traditional fixed pitch propellers changes position by altering the rotational velocity of the individual propellers. In turbulent conditions this method has shown to be insufficient, and achieving stability and precise landings is challenging.\bigskip

%Traditional fixed pitch quadcopters are limited by the speed at which the propellers can change their rotational speed. The rate of change depends on the inertia of the rotating parts and the torque- speed characteristics of the motors. The responsiveness of fixed pitch quadcopters has shown to be inadequate in turbulent and challenging conditions where fast changes of thrust are required.

% A quadcopter with traditional fixed pitch propellers changes position by accelerating or decelerating the individual propellers. Thus, generating more or less thrust as required.
% But in some cases this method has shown to be to slow, and the quadcopter can't be held stable in disturbing and turbulent conditions, especially during landing. \bigskip

For this project the Norwegian Defence Research Establishment(FFI) wants the students to investigate the characteristics and benefits of variable pitch for use on quadcopters. The objectives are:
\begin{enumerate}
    \item Build a small quadcopter ($<2,5 kg$) with variable pitch
    \item Build a small quadcopter ($<2,5 kg$) with fixed pitch
    \item Investigate if variable pitch gives the possibility of more stable flight and landing in turbulent conditions
    \item Investigate if variable pitch improves response time
    \item Compare the fixed pitch quadcopter against the variable pitch quadcopter
\end{enumerate}

\section*{Project Outline} 
In order to answer the questions posed by FFI, a quadcopter must be designed, built and then tested. \bigskip

The team has to acquire a rigorous understanding of quadcopters from theory to practice. Two quadcopters must be built, a control system must be developed, and the control system has to be translated into logic software in a flight controller.\bigskip

Flight testing and data gathering is performed in KONGSBERG Innovation Center (KIC) using a motion capture system called Qualisys. The motion capture system is used to accurately measure and log the position and trajectories of the quadcopter. \bigskip

When a working prototype is produced, there will be performed a series of tests to acquire the necessary data to answer the questions posed by FFI.

\clearpage 

\section*{Motivation}
Building a quadcopter is an interdisciplinary effort, containing elements from software, hardware and electrical engineering. This project is a great opportunity to experience multidisciplinary teamwork and engineering first hand. Taking on such a complicated and interdependent engineering task is appealing to all the team members and a great source of motivation. The fact that no commercially available variable pitch flight controller or quadcopter with RPM control exists, makes the project appealing since we are creating something that is not possible to buy.\bigskip

\section*{Limitations}
There are some limitations to every project and this one is no exception. The scope of this project is at the limit of what can be achieved during a 5 month bachelors project. The subject of this project is very complex and none of the team members have previous experience with making a fully functional quadcopter. There is marginal information about quadcopters with variable pitch available, and there exists only a few commercially available quadcopters with variable pitch. \bigskip

No commercially available flight controller supports variable pitch. This means that the team has to create a new flight controller or do changes to an existing one. The stability of the final quadcopter is of great importance to create the relevant data for answering the objectives given by FFI.  
Ideally, the team should use autopilot software to get the most reproducible test results. Autonomy is a complex and a large subject in itself, and will be implemented only if possible. \bigskip

\section*{Approach}
The project has a finite end date, and the team cannot pay attention to, document or discuss all aspects of quadcopters. The most important aspects considered is the stability of variable pitch quadcopters, especially during landing, and how they compare to traditional fixed pitch quadcopters. \bigskip

In order to answer the questions posed by FFI, two quadcopters must be created. One with variable pitch and another with fixed pitch for comparison. The most important question is: "Can variable pitch provide a more stable landing than fixed pitch?". To test this, multiple landings will be performed, preferably autonomously. The tests will be executed with the same conditions in the same environment to ensure that as many variables as possible are kept constant.\bigskip 

If the quadcopters becomes sufficiently stable, turbulence will be introduced in the test environment. For example by using a leaf-blower. Additionally, to test the differences between the two quadcopter configurations, tests in a testing rig will be performed. The data obtained from static testing will help substantiate the answers to the questions.\bigskip

To handle the complexity and diversity of the project, Scrum was chosen. Scrum ensures that sudden unforeseen changes are identified early and acted upon. It gives transparency within the team and ensures rapid development and good communication between all parties.


%With the two quadcopters the team is building, it should be possible to test differences between the fixed- and variable pitch quadcopter. 

